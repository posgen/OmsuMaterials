% Преамбула
\documentclass[12pt,a4paper]{article}
\usepackage[T2A]{fontenc}
\usepackage[utf8]{inputenc}
\usepackage{graphicx}
\usepackage[russian]{babel}
\usepackage{color}
\usepackage{cite}
\usepackage{bm}

%Параметры страницы
\textwidth 17cm
\textheight 23cm
\oddsidemargin 0cm
\topmargin -1cm
\def\Year{\expandafter\YEAR\the\year}
\def\YEAR#1#2#3#4{#3#4}
\definecolor{grey}{gray}{0.3}

\begin{document}

\begin{titlepage}
\begin{center}
{\large\bf Министерство образования и науки \linebreak российской федерации\\[14pt]
    Государственное образовательное учреждение
    высшего профессионального образования \linebreak
    Омский государственный университет им.~Ф.М.~Достоевского\\[1cm]
    Кафедра теоретической физики
    }\\[4.2cm]
{\large \textbf{Отчет по лабораторной работе}}\\[1cm]
{\Large\bf Компьютерное моделирование равновесных характеристик 2-мерной модели Изинга}\\[3cm]
\end{center}
\hspace*{9cm}{\bf Работу выполнили:}\\
\hspace*{12cm}{\bf И.О.~Фамилия}\\[2pt]
\hspace*{12cm}{\bf <И.О.~Фамилия>}\\[1.6cm]

\begin{center}
{\bf Омск -- 20\Year}
\end{center}
\end{titlepage}

\newpage
\section{Постановка задачи}
Не стесняемся написать полную формулировку проводимой работы из пособия.


\section{Описание модели исследования и условий моделирования}
Обозначается общая модель: гамильтониан, ее представление - 2D или 3D
решётка, граничные условия.

Приводятся формулы для вычисления основных характеристик. Например: в ходе расчетов
вычислялась намагниченность системы изинговских спинов по формуле
\begin{equation}
m = \frac{1}{N} \sum_{i = 1}^{N} S_i
\end{equation}

\section{Результаты работы}
Таблицы, рисунки, графики - все приветствуется.


\section{Выводы}
Резюме того, что стало понятно из проведенной лабораторной работы

\end{document}
